% math packages
\usepackage{mathtools}  % Includes amsmath
\usepackage{amssymb}    % Math symbols such as \mathbb
\usepackage{amsthm}	   % theorem environments
\usepackage{xpatch}    % Patch amsthm \swappedhead to add period after theorem head
\xpatchcmd\swappedhead{~}{.~}{}{}   % Add period after swapped-number theorem head
\usepackage{stmaryrd}   % more symbols
\usepackage{mathrsfs}   % RSFS font in math mode 

% other packages
\usepackage{tikz-cd}
\usepackage{quiver}
\usepackage{graphicx}
\graphicspath{ {../assets/} }
\usepackage{enumitem}
\usepackage{color}
\usepackage{hyperref}
\hypersetup{
    colorlinks=true,
    linktoc=all,     %set to all if you want both sections and subsections linked
    linkcolor=.,	 %the value "." sets the link color to match the surrounding text color
}

% proper inline math display, adjust height for symbols like \sum
\everymath{\displaystyle}

% define tags for math use..

\renewcommand\qedsymbol{$\blacksquare$}


\newtheoremstyle{statement}%
	{}%
	{}%
	{\itshape}%
	{}%
	{\bfseries}%
	{: --- }%
	{ }%
	{}%

\newtheoremstyle{defn}%
	{}%
	{}%
	{\mdseries}%
	{}%
	{\itshape}%
	{.}%
	{ }%
	{}%

\newtheoremstyle{discussion}%
	{}%
	{}%
	{\mdseries}%
	{}%
	{\bfseries}%
	{.}%
	{\newline}%
	{}%

\newtheoremstyle{exercise}
	{}%
	{}%
	{\mdseries}%
	{}%
	{\scshape}%
	{.}%
	{10pt}%
	{}%

\theoremstyle{statement}
\swapnumbers

% Generic Statements

\newtheorem{theorem}{Theorem}[section]
\newtheorem{lemma}[theorem]{Lemma}
\newtheorem{proposition}[theorem]{Proposition}
\newtheorem{corollary}[theorem]{Corollary}
\newtheorem{fact}[theorem]{Fact}
\newtheorem{claim}[theorem]{Claim}

% Discussions and Statements with Custom Names (Discussions are the designated format for examples)

\newcommand{\thistheoremname}{}
\newtheorem{generictheorem}[theorem]{\thistheoremname}
\newenvironment{namedtheorem}[1]
	{\renewcommand{\thistheoremname}{#1}%
	\begin{generictheorem}}
	{\end{generictheorem}}

\theoremstyle{discussion}
\newcommand{\thisdiscussionname}{}
\newtheorem{genericdiscussion}[theorem]{\thisdiscussionname}
\newenvironment{nameddiscussion}[1]
	{\renewcommand{\thisdiscussionname}{#1}%
	\begin{genericdiscussion}}
	{\end{genericdiscussion}}

% Definitions and Remarks

\theoremstyle{definition}
\newtheorem{definition}[theorem]{Definition}
\newtheorem{remark}[theorem]{Remark}

% Exercises

\theoremstyle{exercise}
\newtheorem{exercise}{Exercise}[section]

\theoremstyle{exercise}
\newcommand{\thisexercisename}{}
\newtheorem{genericexercise}[exercise]{\thisexercisename}
\newenvironment{namedexercise}[1]
	{\renewcommand{\thisexercisename}{#1}%
	\begin{genericexercise}}
	{\end{genericexercise}}

\renewcommand*{\theexercise}{\textbf{\thesection.\Alph{exercise}}} % Redefine exercise counter to use letters instead of numbers

% Gives begin{solution} same formating as \begin{proof}

\newenvironment{solution}
  {\begin{proof}[Solution]}
  {\end{proof}}

% Custom Math Operators

% Algebraic Geometry

\DeclareMathOperator{\Spec}{Spec}
\DeclareMathOperator{\Proj}{Proj}
\DeclareMathOperator{\Pic}{Pic}
\DeclareMathOperator{\Div}{div}
\DeclareMathOperator{\CaDiv}{CaDiv}
\DeclareMathOperator{\Cl}{Cl}
\DeclareMathOperator{\CaCl}{CaCl}
\DeclareMathOperator{\SheafHom}{\mathscr{H}\text{\kern -3pt {\calligra\large om}}\;}
\DeclareMathOperator{\SheafExt}{\mathscr{E}\text{\kern -3pt {\calligra\large xt}}\;}
\DeclareMathOperator{\Supp}{Supp}

% Other Algebra

\DeclareMathOperator{\pr}{pr}
\DeclareMathOperator{\nil}{nil}
\DeclareMathOperator{\Hom}{Hom}
\DeclareMathOperator{\codim}{codim}
\DeclareMathOperator{\Aut}{Aut}
\DeclareMathOperator{\End}{End}
\DeclareMathOperator{\colim}{colim}
\DeclareMathOperator{\characteristic}{char}
\DeclareMathOperator{\id}{id}
\DeclareMathOperator{\Span}{Span}
\DeclareMathOperator{\sgn}{sgn}
\DeclareMathOperator{\Tr}{Tr}
\DeclareMathOperator{\N}{N}
\DeclareMathOperator{\im}{im}
\DeclareMathOperator{\coker}{coker}
\DeclareMathOperator{\HH}{H}
\DeclareMathOperator{\hh}{h}
\DeclareMathOperator{\rank}{rank}
\DeclareMathOperator{\acts}{\curvearrowright}
\DeclareMathOperator{\trdeg}{tr. deg}
\DeclareMathOperator{\Tor}{Tor}
\DeclareMathOperator{\Ext}{Ext}
\DeclareMathOperator{\Gal}{Gal}
\DeclareMathOperator{\sep}{sep}
\DeclareMathOperator{\Syz}{Syz}
\DeclareMathOperator{\pd}{pd}
\DeclareMathOperator{\depth}{depth}
\DeclareMathOperator{\bm}{bm}
\DeclareMathOperator{\burch}{burch}
\DeclareMathOperator{\length}{length}
\DeclareMathOperator{\socle}{socle}
\DeclareMathOperator{\Char}{char}

